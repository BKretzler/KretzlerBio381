\documentclass{article}\usepackage[]{graphicx}\usepackage[]{color}
% maxwidth is the original width if it is less than linewidth
% otherwise use linewidth (to make sure the graphics do not exceed the margin)
\makeatletter
\def\maxwidth{ %
  \ifdim\Gin@nat@width>\linewidth
    \linewidth
  \else
    \Gin@nat@width
  \fi
}
\makeatother

\definecolor{fgcolor}{rgb}{0.345, 0.345, 0.345}
\newcommand{\hlnum}[1]{\textcolor[rgb]{0.686,0.059,0.569}{#1}}%
\newcommand{\hlstr}[1]{\textcolor[rgb]{0.192,0.494,0.8}{#1}}%
\newcommand{\hlcom}[1]{\textcolor[rgb]{0.678,0.584,0.686}{\textit{#1}}}%
\newcommand{\hlopt}[1]{\textcolor[rgb]{0,0,0}{#1}}%
\newcommand{\hlstd}[1]{\textcolor[rgb]{0.345,0.345,0.345}{#1}}%
\newcommand{\hlkwa}[1]{\textcolor[rgb]{0.161,0.373,0.58}{\textbf{#1}}}%
\newcommand{\hlkwb}[1]{\textcolor[rgb]{0.69,0.353,0.396}{#1}}%
\newcommand{\hlkwc}[1]{\textcolor[rgb]{0.333,0.667,0.333}{#1}}%
\newcommand{\hlkwd}[1]{\textcolor[rgb]{0.737,0.353,0.396}{\textbf{#1}}}%
\let\hlipl\hlkwb

\usepackage{framed}
\makeatletter
\newenvironment{kframe}{%
 \def\at@end@of@kframe{}%
 \ifinner\ifhmode%
  \def\at@end@of@kframe{\end{minipage}}%
  \begin{minipage}{\columnwidth}%
 \fi\fi%
 \def\FrameCommand##1{\hskip\@totalleftmargin \hskip-\fboxsep
 \colorbox{shadecolor}{##1}\hskip-\fboxsep
     % There is no \\@totalrightmargin, so:
     \hskip-\linewidth \hskip-\@totalleftmargin \hskip\columnwidth}%
 \MakeFramed {\advance\hsize-\width
   \@totalleftmargin\z@ \linewidth\hsize
   \@setminipage}}%
 {\par\unskip\endMakeFramed%
 \at@end@of@kframe}
\makeatother

\definecolor{shadecolor}{rgb}{.97, .97, .97}
\definecolor{messagecolor}{rgb}{0, 0, 0}
\definecolor{warningcolor}{rgb}{1, 0, 1}
\definecolor{errorcolor}{rgb}{1, 0, 0}
\newenvironment{knitrout}{}{} % an empty environment to be redefined in TeX

\usepackage{alltt}
\IfFileExists{upquote.sty}{\usepackage{upquote}}{}
\begin{document}


%Make a comment with percent sign


\section*{Main title}

\subsection*{first subtitle}

\subsubsection*{second subtitle}

%remove the asterisk for numbered titles
\section{Main title}

\subsection{first subtitle}

\subsubsection{second subtitle}

We cannot go below a third level title as above

What will I have for dinner tonight? Most of this follows plain text rules. \LaTeX is amazing


here is \textbf{bold face}\\
here is \textit{italic font}\\
here is \texttt{plain text}\\
double back slash indicates a carriage return instead of double space


\section*{Quotations}

``use two back ticks to start a quotation, and 2 apostrophes to end''

\section*{lists}

must begin the itemized environment with begin command

\begin{itemize}
  \item first bullet point
  \item second
  \item third
  
the end statement

\end{itemize}

or enumerate

\begin{enumerate}
  \item first bullet point
  \item second
  \item third
  
the end statement

\end{enumerate}

enumerate generates numbered list, itemized does bullet points. Can do sublayer but Nick doesn't know this.

\section*{Verbatim environment}

\begin{verbatim}

Text here is literal, no formatting signs.

Also no margin control so need carriage returns

\end{verbatim}


\section*{Quotation environment}

\begin{quote}

Indents an entire paragraph to display an extended quote. You still have to supply ``quotation marks''

\end{quote}

\section*{R code}

use control alt I to put in a code chunk

within <> can put in echo, and other commands

\begin{knitrout}
\definecolor{shadecolor}{rgb}{0.969, 0.969, 0.969}\color{fgcolor}\begin{kframe}
\begin{alltt}
\hlstd{x} \hlkwb{<-} \hlkwd{seq}\hlstd{(}\hlnum{1}\hlstd{,}\hlnum{10}\hlstd{,}\hlnum{2}\hlstd{)}
\hlstd{y} \hlkwb{<-} \hlkwd{seq}\hlstd{(}\hlnum{1}\hlstd{,}\hlnum{12}\hlstd{,}\hlnum{3}\hlstd{)}
\end{alltt}
\end{kframe}
\end{knitrout}


\end{document}
